\chapter*{Összefoglalás}
\addcontentsline{toc}{chapter}{Összefoglalás}  

\Aref{sec:classical_DMP}. részben ismeretetett eredmények alapján az elliptikus Dirichlet-peremmel ellátott feladatok megoldásánál, ha a megoldást lineáris végeselemekkel közelítjük, akkor ismerünk elégséges feltételeket a klasszikus diszkrét maximum-elv teljesülésére. 2 dimenziós esetben például, ha háromszögrácsot alkalmazunk, elegendő biztosítanunk, hogy a rács megfelelő finomságú és egyenletesen hegyesszögű legyen, és ha nincs a feladatban visszacsatolás, akkor a derékszögek is megengedettek. Ezek a feltételek viszonylag könnyen teljesíthetők a számítások során.

\Aref{sec:general_DMP}. szakaszban az 1 dimenziós Poisson-egyenletet vizsgáltuk. Láthattuk, hogy a klasszikus diszkrét maximum-elv nem terjeszthető ki a magasabbrendű végeselemes közelítések esetére, mert a forrásfüggvény előjele ekkor nem határozza meg a diszkrét feladat jobb oldalának előjelét. A probléma kiküszöbölésére bevezettük a diszkrét maximum-elv általánosított alakját, ami a forrásfüggvény helyett annak a végeselemes altérre vett $L^2$-vetületére szab előjelfeltételt. Ezután 1 dimenzióban megmutattuk, hogy bizonyos feltételeket kielégítő kvadratúraformulák létezése elégséges feltétele az általánosított diszkrét maximum-elv teljesülésének magasabbrendű közelítésekre.

Nyitott kérdés, hogy az általánosított diszkrét maximum-elv kiterjeszthető-e magasabb dimenzióra és általánosabb elliptikus feladatokra. Mivel a gyakorlatban van igény a magasabbrendű közelítések alkalmazására, fontos lenne a maximum-elv kiterjeszthetőségének vizsgálata. 

\chapter*{Köszönetnyilvánítás}
\addcontentsline{toc}{chapter}{Köszönetnyilvánítás}

Ezúton köszönetet mondani konzulensemnek, Karátson Jánosnak a segítségéért, türelméért, és hogy munkámat alaposan és kritikusan ellenőrizte. 

Továbbá szeretném megköszönni családomnak és barátaimnak a szakszerű hibaellenőrzést, valamint a rengeteg támogatást, amit képzésem ideje alatt kaptam tőlük.


% \appendix
% \chapter{A futtatásokhoz tartozó MATLAB fájlok}\label{matlabkodok}
\chapter*{Függelék}\label{fuggelek}
\addcontentsline{toc}{chapter}{Függelék}  

\section*{A futtatásokhoz tartozó MATLAB fájlok}\label{matlabkodok}
\addcontentsline{toc}{section}{A futtatásokhoz tartozó MATLAB fájlok} 

\lstinputlisting{\matlabpath/solve_2DPoi_for_multiple_rhs_utf8.m"}
\lstinputlisting{\matlabpath/FEM_2DPoissonHomogenousDirichlet_utf8.m"}



