\begin{abstract}

A parciális differenciálegyenletekkel leírható folyamatokhoz kapcsolódó legfontosabb kvalitatív tulajdonságok a  maximum-elvek, illetve ezek egyik következménye, a  nemnegativitási tulajdonság. Mivel a legtöbb differenciálegyenlet csak numerikusan oldható meg, a numerikus modellektől is elvárjuk a folytonos modell kvalitatív tulajdonságaival ekvivalens tulajdonságok teljesülését. A dolgozatban a lineáris másodrendű elliptikus feladatokkal foglalkozunk, Dirichlet-peremfeltétel mellett.  Először ismertetjük a Dirichlet-feladat végeselemes diszkretizációjának elméleti alapjait, ezután  rátérünk a vizsgált elliptikus feladatokra igazolható maximum-elvekre. Bemutatjuk a folytonos maximum-elvet, és annak következményeit, majd definiáljuk a klasszikus diszkrét maximum-elvet, ami lineáris végeselemes közelítésekre alkalmazható, azonban magasabbrendű közelítésekre nem terjeszthető ki. Ezután kimondjuk az általánosított diszkrét maximum-elvet, illetve annak egy speciális változatát, a nemnegativitási elvet, homogén Dirichlet-peremmel adott 1D Poisson-egyenlet végeselemes megoldására, és elégséges feltételt mutatunk a magasabbrendű végeselemes approximáció esetére. Végül néhány példán keresztül szemléltetjük a nemnegativitási tulajdonságot lineáris végeselemes közelítésre.

\end{abstract}